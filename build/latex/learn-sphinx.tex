%% Generated by Sphinx.
\def\sphinxdocclass{report}
\documentclass[letterpaper,10pt,english]{sphinxmanual}
\ifdefined\pdfpxdimen
   \let\sphinxpxdimen\pdfpxdimen\else\newdimen\sphinxpxdimen
\fi \sphinxpxdimen=.75bp\relax
\ifdefined\pdfimageresolution
    \pdfimageresolution= \numexpr \dimexpr1in\relax/\sphinxpxdimen\relax
\fi
%% let collapsible pdf bookmarks panel have high depth per default
\PassOptionsToPackage{bookmarksdepth=5}{hyperref}

\PassOptionsToPackage{warn}{textcomp}
\usepackage[utf8]{inputenc}
\ifdefined\DeclareUnicodeCharacter
% support both utf8 and utf8x syntaxes
  \ifdefined\DeclareUnicodeCharacterAsOptional
    \def\sphinxDUC#1{\DeclareUnicodeCharacter{"#1}}
  \else
    \let\sphinxDUC\DeclareUnicodeCharacter
  \fi
  \sphinxDUC{00A0}{\nobreakspace}
  \sphinxDUC{2500}{\sphinxunichar{2500}}
  \sphinxDUC{2502}{\sphinxunichar{2502}}
  \sphinxDUC{2514}{\sphinxunichar{2514}}
  \sphinxDUC{251C}{\sphinxunichar{251C}}
  \sphinxDUC{2572}{\textbackslash}
\fi
\usepackage{cmap}
\usepackage[T1]{fontenc}
\usepackage{amsmath,amssymb,amstext}
\usepackage{babel}



\usepackage{tgtermes}
\usepackage{tgheros}
\renewcommand{\ttdefault}{txtt}



\usepackage[Bjarne]{fncychap}
\usepackage{sphinx}

\fvset{fontsize=auto}
\usepackage{geometry}


% Include hyperref last.
\usepackage{hyperref}
% Fix anchor placement for figures with captions.
\usepackage{hypcap}% it must be loaded after hyperref.
% Set up styles of URL: it should be placed after hyperref.
\urlstyle{same}

\addto\captionsenglish{\renewcommand{\contentsname}{Contents:}}

\usepackage{sphinxmessages}
\setcounter{tocdepth}{1}



\title{learn\sphinxhyphen{}sphinx}
\date{Jul 20, 2022}
\release{1.0.0rc1}
\author{gzh}
\newcommand{\sphinxlogo}{\vbox{}}
\renewcommand{\releasename}{Release}
\makeindex
\begin{document}

\pagestyle{empty}
\sphinxmaketitle
\pagestyle{plain}
\sphinxtableofcontents
\pagestyle{normal}
\phantomsection\label{\detokenize{index::doc}}



\chapter{Getting started}
\label{\detokenize{chapter1:getting-started}}\label{\detokenize{chapter1:id1}}\label{\detokenize{chapter1::doc}}

\section{Installing your doc directory}
\label{\detokenize{chapter1:installing-your-doc-directory}}\label{\detokenize{chapter1:installing-docdir}}
\sphinxAtStartPar
You may already have sphinx \sphinxhref{http://sphinx.pocoo.org/}{sphinx}
installed \textendash{} you can check by doing:

\begin{sphinxVerbatim}[commandchars=\\\{\}]
\PYG{n}{python} \PYG{o}{\PYGZhy{}}\PYG{n}{c} \PYG{l+s+s1}{\PYGZsq{}}\PYG{l+s+s1}{import sphinx}\PYG{l+s+s1}{\PYGZsq{}}
\end{sphinxVerbatim}

\sphinxAtStartPar
If that fails grab the latest version of and install it with:

\begin{sphinxVerbatim}[commandchars=\\\{\}]
\PYG{o}{\PYGZgt{}} \PYG{n}{sudo} \PYG{n}{easy\PYGZus{}install} \PYG{o}{\PYGZhy{}}\PYG{n}{U} \PYG{n}{Sphinx}
\end{sphinxVerbatim}

\sphinxAtStartPar
Now you are ready to build a template for your docs, using
sphinx\sphinxhyphen{}quickstart:

\begin{sphinxVerbatim}[commandchars=\\\{\}]
\PYG{o}{\PYGZgt{}} \PYG{n}{sphinx}\PYG{o}{\PYGZhy{}}\PYG{n}{quickstart}
\end{sphinxVerbatim}

\sphinxAtStartPar
accepting most of the defaults.  I choose “sampledoc” as the name of my
project.  cd into your new directory and check the contents:

\begin{sphinxVerbatim}[commandchars=\\\{\}]
\PYG{n}{home}\PYG{p}{:}\PYG{o}{\PYGZti{}}\PYG{o}{/}\PYG{n}{tmp}\PYG{o}{/}\PYG{n}{sampledoc}\PYG{o}{\PYGZgt{}} \PYG{n}{ls}
\PYG{n}{Makefile}      \PYG{n}{\PYGZus{}static}         \PYG{n}{conf}\PYG{o}{.}\PYG{n}{py}
\PYG{n}{build}         \PYG{n}{\PYGZus{}templates}      \PYG{n}{index}\PYG{o}{.}\PYG{n}{rst}
\end{sphinxVerbatim}

\sphinxAtStartPar
The index.rst is the master ReST for your project, but before adding
anything, let’s see if we can build some html:

\begin{sphinxVerbatim}[commandchars=\\\{\}]
\PYG{n}{make} \PYG{n}{html}
\end{sphinxVerbatim}

\sphinxAtStartPar
If you now point your browser to \sphinxcode{\sphinxupquote{build/html/index.html}}, you
should see a basic sphinx site.

\noindent\sphinxincludegraphics{{basic_screenshot}.png}


\subsection{Fetching the data}
\label{\detokenize{chapter1:fetching-the-data}}\label{\detokenize{chapter1:id2}}
\sphinxAtStartPar
Now we will start to customize out docs.  Grab a couple of files from
the \sphinxhref{https://github.com/matplotlib/sampledoc}{web site}
or git.  You will need \sphinxcode{\sphinxupquote{getting\_started.rst}} and
\sphinxcode{\sphinxupquote{images/basic\_screenshot.png}}.  All of the files live in the
“completed” version of this tutorial, but since this is a tutorial,
we’ll just grab them one at a time, so you can learn what needs to be
changed where.  Since we have more files to come, I’m going to grab
the whole git directory and just copy the files I need over for now.
First, I’ll cd up back into the directory containing my project, check
out the “finished” product from git, and then copy in just the files I
need into my \sphinxcode{\sphinxupquote{sampledoc}} directory:

\begin{sphinxVerbatim}[commandchars=\\\{\}]
\PYG{n}{home}\PYG{p}{:}\PYG{o}{\PYGZti{}}\PYG{o}{/}\PYG{n}{tmp}\PYG{o}{/}\PYG{n}{sampledoc}\PYG{o}{\PYGZgt{}} \PYG{n}{pwd}
\PYG{o}{/}\PYG{n}{Users}\PYG{o}{/}\PYG{n}{jdhunter}\PYG{o}{/}\PYG{n}{tmp}\PYG{o}{/}\PYG{n}{sampledoc}
\PYG{n}{home}\PYG{p}{:}\PYG{o}{\PYGZti{}}\PYG{o}{/}\PYG{n}{tmp}\PYG{o}{/}\PYG{n}{sampledoc}\PYG{o}{\PYGZgt{}} \PYG{n}{cd} \PYG{o}{.}\PYG{o}{.}
\PYG{n}{home}\PYG{p}{:}\PYG{o}{\PYGZti{}}\PYG{o}{/}\PYG{n}{tmp}\PYG{o}{\PYGZgt{}} \PYG{n}{git} \PYG{n}{clone} \PYG{n}{https}\PYG{p}{:}\PYG{o}{/}\PYG{o}{/}\PYG{n}{github}\PYG{o}{.}\PYG{n}{com}\PYG{o}{/}\PYG{n}{matplotlib}\PYG{o}{/}\PYG{n}{sampledoc}\PYG{o}{.}\PYG{n}{git} \PYG{n}{tutorial}
\PYG{n}{Cloning} \PYG{n}{into} \PYG{l+s+s1}{\PYGZsq{}}\PYG{l+s+s1}{tutorial}\PYG{l+s+s1}{\PYGZsq{}}\PYG{o}{.}\PYG{o}{.}\PYG{o}{.}
\PYG{n}{remote}\PYG{p}{:} \PYG{n}{Counting} \PYG{n}{objects}\PYG{p}{:} \PYG{l+m+mi}{87}\PYG{p}{,} \PYG{n}{done}\PYG{o}{.}
\PYG{n}{remote}\PYG{p}{:} \PYG{n}{Compressing} \PYG{n}{objects}\PYG{p}{:} \PYG{l+m+mi}{100}\PYG{o}{\PYGZpc{}} \PYG{p}{(}\PYG{l+m+mi}{43}\PYG{o}{/}\PYG{l+m+mi}{43}\PYG{p}{)}\PYG{p}{,} \PYG{n}{done}\PYG{o}{.}
\PYG{n}{remote}\PYG{p}{:} \PYG{n}{Total} \PYG{l+m+mi}{87} \PYG{p}{(}\PYG{n}{delta} \PYG{l+m+mi}{45}\PYG{p}{)}\PYG{p}{,} \PYG{n}{reused} \PYG{l+m+mi}{83} \PYG{p}{(}\PYG{n}{delta} \PYG{l+m+mi}{41}\PYG{p}{)}
\PYG{n}{Unpacking} \PYG{n}{objects}\PYG{p}{:} \PYG{l+m+mi}{100}\PYG{o}{\PYGZpc{}} \PYG{p}{(}\PYG{l+m+mi}{87}\PYG{o}{/}\PYG{l+m+mi}{87}\PYG{p}{)}\PYG{p}{,} \PYG{n}{done}\PYG{o}{.}
\PYG{n}{Checking} \PYG{n}{connectivity}\PYG{o}{.}\PYG{o}{.}\PYG{o}{.} \PYG{n}{done}
\PYG{n}{home}\PYG{p}{:}\PYG{o}{\PYGZti{}}\PYG{o}{/}\PYG{n}{tmp}\PYG{o}{\PYGZgt{}} \PYG{n}{cp} \PYG{n}{tutorial}\PYG{o}{/}\PYG{n}{getting\PYGZus{}started}\PYG{o}{.}\PYG{n}{rst} \PYG{n}{sampledoc}\PYG{o}{/}
\PYG{n}{home}\PYG{p}{:}\PYG{o}{\PYGZti{}}\PYG{o}{/}\PYG{n}{tmp}\PYG{o}{\PYGZgt{}} \PYG{n}{cp} \PYG{n}{tutorial}\PYG{o}{/}\PYG{n}{\PYGZus{}static}\PYG{o}{/}\PYG{n}{images}\PYG{o}{/}\PYG{n}{basic\PYGZus{}screenshot}\PYG{o}{.}\PYG{n}{png} \PYG{n}{sampledoc}\PYG{o}{/}\PYG{n}{\PYGZus{}static}\PYG{o}{/}
\end{sphinxVerbatim}

\sphinxAtStartPar
The last step is to modify \sphinxcode{\sphinxupquote{index.rst}} to include the
\sphinxcode{\sphinxupquote{getting\_started.rst}} file (be careful with the indentation, the
“g” in “getting\_started” should line up with the ‘:’ in \sphinxcode{\sphinxupquote{:maxdepth}}:

\begin{sphinxVerbatim}[commandchars=\\\{\}]
\PYG{n}{Contents}\PYG{p}{:}

\PYG{o}{.}\PYG{o}{.} \PYG{n}{toctree}\PYG{p}{:}\PYG{p}{:}
   \PYG{p}{:}\PYG{n}{maxdepth}\PYG{p}{:} \PYG{l+m+mi}{2}

   \PYG{n}{getting\PYGZus{}started}\PYG{o}{.}\PYG{n}{rst}
\end{sphinxVerbatim}

\sphinxAtStartPar
and then rebuild the docs:

\begin{sphinxVerbatim}[commandchars=\\\{\}]
\PYG{n}{cd} \PYG{n}{sampledoc}
\PYG{n}{make} \PYG{n}{html}
\end{sphinxVerbatim}

\sphinxAtStartPar
When you reload the page by refreshing your browser pointing to
\sphinxcode{\sphinxupquote{build/html/index.html}}, you should see a link to the
“Getting Started” docs, and in there this page with the screenshot.
\sphinxtitleref{Voila!}

\sphinxAtStartPar
We can also use the image directive in \sphinxcode{\sphinxupquote{index.rst}} to include to the screenshot above
with:

\begin{sphinxVerbatim}[commandchars=\\\{\}]
\PYG{o}{.}\PYG{o}{.} \PYG{n}{image}\PYG{p}{:}\PYG{p}{:}
   \PYG{n}{images}\PYG{o}{/}\PYG{n}{basic\PYGZus{}screenshot}\PYG{o}{.}\PYG{n}{png}
\end{sphinxVerbatim}

\sphinxAtStartPar
Next we’ll customize the look and feel of our site to give it a logo,
some custom css, and update the navigation panels to look more like
the \sphinxhref{http://sphinx.pocoo.org/}{sphinx} site itself \textendash{} see
\DUrole{xref,std,std-ref}{custom\_look}.


\chapter{Sphinx extensions for embedded plots, math and more}
\label{\detokenize{chapter2:sphinx-extensions-for-embedded-plots-math-and-more}}\label{\detokenize{chapter2:extensions}}\label{\detokenize{chapter2::doc}}
\sphinxAtStartPar
Sphinx is written in python, and supports the ability to write custom
extensions.  We’ve written a few for the matplotlib documentation,
some of which are part of matplotlib itself in the
matplotlib.sphinxext module, some of which are included only in the
sphinx doc directory, and there are other extensions written by other
groups, eg numpy and ipython.  We’re collecting these in this tutorial
and showing you how to install and use them for your own project.
First let’s grab the python extension files from the \sphinxcode{\sphinxupquote{sphinxext}}
directory from git (see {\hyperref[\detokenize{chapter1:fetching-the-data}]{\sphinxcrossref{\DUrole{std,std-ref}{Fetching the data}}}}), and install them in
our \sphinxcode{\sphinxupquote{sampledoc}} project \sphinxcode{\sphinxupquote{sphinxext}} directory:

\begin{sphinxVerbatim}[commandchars=\\\{\}]
\PYG{n}{home}\PYG{p}{:}\PYG{o}{\PYGZti{}}\PYG{o}{/}\PYG{n}{tmp}\PYG{o}{/}\PYG{n}{sampledoc}\PYG{o}{\PYGZgt{}} \PYG{n}{mkdir} \PYG{n}{sphinxext}
\PYG{n}{home}\PYG{p}{:}\PYG{o}{\PYGZti{}}\PYG{o}{/}\PYG{n}{tmp}\PYG{o}{/}\PYG{n}{sampledoc}\PYG{o}{\PYGZgt{}} \PYG{n}{cp} \PYG{o}{.}\PYG{o}{.}\PYG{o}{/}\PYG{n}{sampledoc\PYGZus{}tut}\PYG{o}{/}\PYG{n}{sphinxext}\PYG{o}{/}\PYG{o}{*}\PYG{o}{.}\PYG{n}{py} \PYG{n}{sphinxext}\PYG{o}{/}
\PYG{n}{home}\PYG{p}{:}\PYG{o}{\PYGZti{}}\PYG{o}{/}\PYG{n}{tmp}\PYG{o}{/}\PYG{n}{sampledoc}\PYG{o}{\PYGZgt{}} \PYG{n}{ls} \PYG{n}{sphinxext}\PYG{o}{/}
\PYG{n}{apigen}\PYG{o}{.}\PYG{n}{py}  \PYG{n}{docscrape}\PYG{o}{.}\PYG{n}{py}  \PYG{n}{docscrape\PYGZus{}sphinx}\PYG{o}{.}\PYG{n}{py}  \PYG{n}{numpydoc}\PYG{o}{.}\PYG{n}{py}
\end{sphinxVerbatim}

\sphinxAtStartPar
In addition to the builtin matplotlib extensions for embedding pyplot
plots and rendering math with matplotlib’s native math engine, we also
have extensions for syntax highlighting ipython sessions, making
inhertiance diagrams, and more.

\sphinxAtStartPar
We need to inform sphinx of our new extensions in the \sphinxcode{\sphinxupquote{conf.py}}
file by adding the following.  First we tell it where to find the extensions:

\begin{sphinxVerbatim}[commandchars=\\\{\}]
\PYG{c+c1}{\PYGZsh{} If your extensions are in another directory, add it here. If the}
\PYG{c+c1}{\PYGZsh{} directory is relative to the documentation root, use}
\PYG{c+c1}{\PYGZsh{} os.path.abspath to make it absolute, like shown here.}
\PYG{n}{sys}\PYG{o}{.}\PYG{n}{path}\PYG{o}{.}\PYG{n}{append}\PYG{p}{(}\PYG{n}{os}\PYG{o}{.}\PYG{n}{path}\PYG{o}{.}\PYG{n}{abspath}\PYG{p}{(}\PYG{l+s+s1}{\PYGZsq{}}\PYG{l+s+s1}{sphinxext}\PYG{l+s+s1}{\PYGZsq{}}\PYG{p}{)}\PYG{p}{)}
\end{sphinxVerbatim}

\sphinxAtStartPar
And then we tell it what extensions to load:

\begin{sphinxVerbatim}[commandchars=\\\{\}]
\PYG{c+c1}{\PYGZsh{} Add any Sphinx extension module names here, as strings. They can be extensions}
\PYG{c+c1}{\PYGZsh{} coming with Sphinx (named \PYGZsq{}sphinx.ext.*\PYGZsq{}) or your custom ones.}
\PYG{n}{extensions} \PYG{o}{=} \PYG{p}{[}\PYG{l+s+s1}{\PYGZsq{}}\PYG{l+s+s1}{matplotlib.sphinxext.only\PYGZus{}directives}\PYG{l+s+s1}{\PYGZsq{}}\PYG{p}{,}
              \PYG{l+s+s1}{\PYGZsq{}}\PYG{l+s+s1}{matplotlib.sphinxext.plot\PYGZus{}directive}\PYG{l+s+s1}{\PYGZsq{}}\PYG{p}{,}
              \PYG{l+s+s1}{\PYGZsq{}}\PYG{l+s+s1}{IPython.sphinxext.ipython\PYGZus{}directive}\PYG{l+s+s1}{\PYGZsq{}}\PYG{p}{,}
              \PYG{l+s+s1}{\PYGZsq{}}\PYG{l+s+s1}{IPython.sphinxext.ipython\PYGZus{}console\PYGZus{}highlighting}\PYG{l+s+s1}{\PYGZsq{}}\PYG{p}{,}
              \PYG{l+s+s1}{\PYGZsq{}}\PYG{l+s+s1}{sphinx.ext.mathjax}\PYG{l+s+s1}{\PYGZsq{}}\PYG{p}{,}
              \PYG{l+s+s1}{\PYGZsq{}}\PYG{l+s+s1}{sphinx.ext.autodoc}\PYG{l+s+s1}{\PYGZsq{}}\PYG{p}{,}
              \PYG{l+s+s1}{\PYGZsq{}}\PYG{l+s+s1}{sphinx.ext.doctest}\PYG{l+s+s1}{\PYGZsq{}}\PYG{p}{,}
              \PYG{l+s+s1}{\PYGZsq{}}\PYG{l+s+s1}{sphinx.ext.inheritance\PYGZus{}diagram}\PYG{l+s+s1}{\PYGZsq{}}\PYG{p}{,}
              \PYG{l+s+s1}{\PYGZsq{}}\PYG{l+s+s1}{numpydoc}\PYG{l+s+s1}{\PYGZsq{}}\PYG{p}{]}
\end{sphinxVerbatim}

\sphinxAtStartPar
Now let’s look at some of these in action.  You can see the literal
source for this file at {\hyperref[\detokenize{chapter2:extensions-literal}]{\sphinxcrossref{\DUrole{std,std-ref}{This file}}}}.


\section{ipython sessions}
\label{\detokenize{chapter2:ipython-sessions}}\label{\detokenize{chapter2:ipython-highlighting}}
\sphinxAtStartPar
Michael Droettboom contributed a sphinx extension which does \sphinxhref{http://pygments.org}{pygments} syntax highlighting on \sphinxhref{http://ipython.scipy.org}{ipython} sessions.  Just use ipython as the
language in the \sphinxcode{\sphinxupquote{sourcecode}} directive:

\begin{sphinxVerbatim}[commandchars=\\\{\}]
\PYG{o}{.}\PYG{o}{.} \PYG{n}{sourcecode}\PYG{p}{:}\PYG{p}{:} \PYG{n}{ipython}

    \PYG{n}{In} \PYG{p}{[}\PYG{l+m+mi}{69}\PYG{p}{]}\PYG{p}{:} \PYG{n}{lines} \PYG{o}{=} \PYG{n}{plot}\PYG{p}{(}\PYG{p}{[}\PYG{l+m+mi}{1}\PYG{p}{,}\PYG{l+m+mi}{2}\PYG{p}{,}\PYG{l+m+mi}{3}\PYG{p}{]}\PYG{p}{)}

    \PYG{n}{In} \PYG{p}{[}\PYG{l+m+mi}{70}\PYG{p}{]}\PYG{p}{:} \PYG{n}{setp}\PYG{p}{(}\PYG{n}{lines}\PYG{p}{)}
      \PYG{n}{alpha}\PYG{p}{:} \PYG{n+nb}{float}
      \PYG{n}{animated}\PYG{p}{:} \PYG{p}{[}\PYG{k+kc}{True} \PYG{o}{|} \PYG{k+kc}{False}\PYG{p}{]}
      \PYG{n}{antialiased} \PYG{o+ow}{or} \PYG{n}{aa}\PYG{p}{:} \PYG{p}{[}\PYG{k+kc}{True} \PYG{o}{|} \PYG{k+kc}{False}\PYG{p}{]}
      \PYG{o}{.}\PYG{o}{.}\PYG{o}{.}\PYG{n}{snip}
\end{sphinxVerbatim}

\sphinxAtStartPar
and you will get the syntax highlighted output below.

\begin{sphinxVerbatim}[commandchars=\\\{\}]
\PYG{n}{In} \PYG{p}{[}\PYG{l+m+mi}{69}\PYG{p}{]}\PYG{p}{:} \PYG{n}{lines} \PYG{o}{=} \PYG{n}{plot}\PYG{p}{(}\PYG{p}{[}\PYG{l+m+mi}{1}\PYG{p}{,}\PYG{l+m+mi}{2}\PYG{p}{,}\PYG{l+m+mi}{3}\PYG{p}{]}\PYG{p}{)}

\PYG{n}{In} \PYG{p}{[}\PYG{l+m+mi}{70}\PYG{p}{]}\PYG{p}{:} \PYG{n}{setp}\PYG{p}{(}\PYG{n}{lines}\PYG{p}{)}
  \PYG{n}{alpha}\PYG{p}{:} \PYG{n+nb}{float}
  \PYG{n}{animated}\PYG{p}{:} \PYG{p}{[}\PYG{k+kc}{True} \PYG{o}{|} \PYG{k+kc}{False}\PYG{p}{]}
  \PYG{n}{antialiased} \PYG{o+ow}{or} \PYG{n}{aa}\PYG{p}{:} \PYG{p}{[}\PYG{k+kc}{True} \PYG{o}{|} \PYG{k+kc}{False}\PYG{p}{]}
  \PYG{o}{.}\PYG{o}{.}\PYG{o}{.}\PYG{n}{snip}
\end{sphinxVerbatim}

\sphinxAtStartPar
This support is included in this template, but will also be included
in a future version of Pygments by default.


\section{Using math}
\label{\detokenize{chapter2:using-math}}\label{\detokenize{chapter2:id1}}
\sphinxAtStartPar
In sphinx you can include inline math \(x\leftarrow y\ x\forall
y\ x-y\) or display math
\begin{equation*}
\begin{split}W^{3\beta}_{\delta_1 \rho_1 \sigma_2} = U^{3\beta}_{\delta_1 \rho_1} + \frac{1}{8 \pi 2} \int^{\alpha_2}_{\alpha_2} d \alpha^\prime_2 \left[\frac{ U^{2\beta}_{\delta_1 \rho_1} - \alpha^\prime_2U^{1\beta}_{\rho_1 \sigma_2} }{U^{0\beta}_{\rho_1 \sigma_2}}\right]\end{split}
\end{equation*}
\sphinxAtStartPar
To include math in your document, just use the math directive; here is
a simpler equation:

\begin{sphinxVerbatim}[commandchars=\\\{\}]
\PYG{o}{.}\PYG{o}{.} \PYG{n}{math}\PYG{p}{:}\PYG{p}{:}

  \PYG{n}{W}\PYG{o}{\PYGZca{}}\PYG{p}{\PYGZob{}}\PYG{l+m+mi}{3}\PYGZbs{}\PYG{n}{beta}\PYG{p}{\PYGZcb{}}\PYG{n}{\PYGZus{}}\PYG{p}{\PYGZob{}}\PYGZbs{}\PYG{n}{delta\PYGZus{}1} \PYGZbs{}\PYG{n}{rho\PYGZus{}1} \PYGZbs{}\PYG{n}{sigma\PYGZus{}2}\PYG{p}{\PYGZcb{}} \PYGZbs{}\PYG{n}{approx} \PYG{n}{U}\PYG{o}{\PYGZca{}}\PYG{p}{\PYGZob{}}\PYG{l+m+mi}{3}\PYGZbs{}\PYG{n}{beta}\PYG{p}{\PYGZcb{}}\PYG{n}{\PYGZus{}}\PYG{p}{\PYGZob{}}\PYGZbs{}\PYG{n}{delta\PYGZus{}1} \PYGZbs{}\PYG{n}{rho\PYGZus{}1}\PYG{p}{\PYGZcb{}}
\end{sphinxVerbatim}

\sphinxAtStartPar
which is rendered as
\begin{equation*}
\begin{split}W^{3\beta}_{\delta_1 \rho_1 \sigma_2} \approx U^{3\beta}_{\delta_1 \rho_1}\end{split}
\end{equation*}
\sphinxAtStartPar
Recent versions of Sphinx include built\sphinxhyphen{}in support for math.
There are three flavors:
\begin{itemize}
\item {} 
\sphinxAtStartPar
sphinx.ext.imgmath: uses dvipng to render the equation

\item {} 
\sphinxAtStartPar
sphinx.ext.mathjax: renders the math in the browser using Javascript

\item {} 
\sphinxAtStartPar
sphinx.ext.jsmath: it’s an older code, but it checks out

\end{itemize}

\sphinxAtStartPar
Additionally, matplotlib has its own math support:
\begin{itemize}
\item {} 
\sphinxAtStartPar
matplotlib.sphinxext.mathmpl

\end{itemize}

\sphinxAtStartPar
See the matplotlib \sphinxhref{https://matplotlib.org/users/mathtext.html}{mathtext guide} for lots
more information on writing mathematical expressions in matplotlib.


\section{Inserting matplotlib plots}
\label{\detokenize{chapter2:inserting-matplotlib-plots}}\label{\detokenize{chapter2:pyplots}}
\sphinxAtStartPar
Inserting automatically\sphinxhyphen{}generated plots is easy.  Simply put the
script to generate the plot in the \sphinxcode{\sphinxupquote{pyplots}} directory, and
refer to it using the \sphinxcode{\sphinxupquote{plot}} directive.  First make a
\sphinxcode{\sphinxupquote{pyplots}} directory at the top level of your project (next to
:\sphinxcode{\sphinxupquote{conf.py}}) and copy the \sphinxcode{\sphinxupquote{ellipses.py\textasciigrave{}}} file into it:

\begin{sphinxVerbatim}[commandchars=\\\{\}]
\PYG{n}{home}\PYG{p}{:}\PYG{o}{\PYGZti{}}\PYG{o}{/}\PYG{n}{tmp}\PYG{o}{/}\PYG{n}{sampledoc}\PYG{o}{\PYGZgt{}} \PYG{n}{mkdir} \PYG{n}{pyplots}
\PYG{n}{home}\PYG{p}{:}\PYG{o}{\PYGZti{}}\PYG{o}{/}\PYG{n}{tmp}\PYG{o}{/}\PYG{n}{sampledoc}\PYG{o}{\PYGZgt{}} \PYG{n}{cp} \PYG{o}{.}\PYG{o}{.}\PYG{o}{/}\PYG{n}{sampledoc\PYGZus{}tut}\PYG{o}{/}\PYG{n}{pyplots}\PYG{o}{/}\PYG{n}{ellipses}\PYG{o}{.}\PYG{n}{py} \PYG{n}{pyplots}\PYG{o}{/}
\end{sphinxVerbatim}

\sphinxAtStartPar
You can refer to this file in your sphinx documentation; by default it
will just inline the plot with links to the source and PF and high
resolution PNGS.  To also include the source code for the plot in the
document, pass the \sphinxcode{\sphinxupquote{include\sphinxhyphen{}source}} parameter:

\begin{sphinxVerbatim}[commandchars=\\\{\}]
\PYG{o}{.}\PYG{o}{.} \PYG{n}{plot}\PYG{p}{:}\PYG{p}{:} \PYG{n}{pyplots}\PYG{o}{/}\PYG{n}{ellipses}\PYG{o}{.}\PYG{n}{py}
   \PYG{p}{:}\PYG{n}{include}\PYG{o}{\PYGZhy{}}\PYG{n}{source}\PYG{p}{:}
\end{sphinxVerbatim}

\sphinxAtStartPar
In the HTML version of the document, the plot includes links to the
original source code, a high\sphinxhyphen{}resolution PNG and a PDF.  In the PDF
version of the document, the plot is included as a scalable PDF.
\begin{description}
\item[{//.. plot:: pyplots/ellipses.py}] \leavevmode\begin{quote}\begin{description}
\item[{include\sphinxhyphen{}source}] \leavevmode
\end{description}\end{quote}

\end{description}

\sphinxAtStartPar
You can also inline code for plots directly, and the code will be
executed at documentation build time and the figure inserted into your
docs; the following code:

\begin{sphinxVerbatim}[commandchars=\\\{\}]
\PYG{o}{.}\PYG{o}{.} \PYG{n}{plot}\PYG{p}{:}\PYG{p}{:}

   \PYG{k+kn}{import} \PYG{n+nn}{matplotlib}\PYG{n+nn}{.}\PYG{n+nn}{pyplot} \PYG{k}{as} \PYG{n+nn}{plt}
   \PYG{k+kn}{import} \PYG{n+nn}{numpy} \PYG{k}{as} \PYG{n+nn}{np}
   \PYG{n}{x} \PYG{o}{=} \PYG{n}{np}\PYG{o}{.}\PYG{n}{random}\PYG{o}{.}\PYG{n}{randn}\PYG{p}{(}\PYG{l+m+mi}{1000}\PYG{p}{)}
   \PYG{n}{plt}\PYG{o}{.}\PYG{n}{hist}\PYG{p}{(} \PYG{n}{x}\PYG{p}{,} \PYG{l+m+mi}{20}\PYG{p}{)}
   \PYG{n}{plt}\PYG{o}{.}\PYG{n}{grid}\PYG{p}{(}\PYG{p}{)}
   \PYG{n}{plt}\PYG{o}{.}\PYG{n}{title}\PYG{p}{(}\PYG{l+s+sa}{r}\PYG{l+s+s1}{\PYGZsq{}}\PYG{l+s+s1}{Normal: \PYGZdl{}}\PYG{l+s+s1}{\PYGZbs{}}\PYG{l+s+s1}{mu=}\PYG{l+s+si}{\PYGZpc{}.2f}\PYG{l+s+s1}{, }\PYG{l+s+s1}{\PYGZbs{}}\PYG{l+s+s1}{sigma=}\PYG{l+s+si}{\PYGZpc{}.2f}\PYG{l+s+s1}{\PYGZdl{}}\PYG{l+s+s1}{\PYGZsq{}}\PYG{o}{\PYGZpc{}}\PYG{p}{(}\PYG{n}{x}\PYG{o}{.}\PYG{n}{mean}\PYG{p}{(}\PYG{p}{)}\PYG{p}{,} \PYG{n}{x}\PYG{o}{.}\PYG{n}{std}\PYG{p}{(}\PYG{p}{)}\PYG{p}{)}\PYG{p}{)}
   \PYG{n}{plt}\PYG{o}{.}\PYG{n}{show}\PYG{p}{(}\PYG{p}{)}
\end{sphinxVerbatim}

\sphinxAtStartPar
produces this output:

\sphinxAtStartPar
//.. plot:

\begin{sphinxVerbatim}[commandchars=\\\{\}]
\PYG{k+kn}{import} \PYG{n+nn}{matplotlib}\PYG{n+nn}{.}\PYG{n+nn}{pyplot} \PYG{k}{as} \PYG{n+nn}{plt}
\PYG{k+kn}{import} \PYG{n+nn}{numpy} \PYG{k}{as} \PYG{n+nn}{np}
\PYG{n}{x} \PYG{o}{=} \PYG{n}{np}\PYG{o}{.}\PYG{n}{random}\PYG{o}{.}\PYG{n}{randn}\PYG{p}{(}\PYG{l+m+mi}{1000}\PYG{p}{)}
\PYG{n}{plt}\PYG{o}{.}\PYG{n}{hist}\PYG{p}{(} \PYG{n}{x}\PYG{p}{,} \PYG{l+m+mi}{20}\PYG{p}{)}
\PYG{n}{plt}\PYG{o}{.}\PYG{n}{grid}\PYG{p}{(}\PYG{p}{)}
\PYG{n}{plt}\PYG{o}{.}\PYG{n}{title}\PYG{p}{(}\PYG{l+s+sa}{r}\PYG{l+s+s1}{\PYGZsq{}}\PYG{l+s+s1}{Normal: \PYGZdl{}}\PYG{l+s+s1}{\PYGZbs{}}\PYG{l+s+s1}{mu=}\PYG{l+s+si}{\PYGZpc{}.2f}\PYG{l+s+s1}{, }\PYG{l+s+s1}{\PYGZbs{}}\PYG{l+s+s1}{sigma=}\PYG{l+s+si}{\PYGZpc{}.2f}\PYG{l+s+s1}{\PYGZdl{}}\PYG{l+s+s1}{\PYGZsq{}}\PYG{o}{\PYGZpc{}}\PYG{p}{(}\PYG{n}{x}\PYG{o}{.}\PYG{n}{mean}\PYG{p}{(}\PYG{p}{)}\PYG{p}{,} \PYG{n}{x}\PYG{o}{.}\PYG{n}{std}\PYG{p}{(}\PYG{p}{)}\PYG{p}{)}\PYG{p}{)}
\PYG{n}{plt}\PYG{o}{.}\PYG{n}{show}\PYG{p}{(}\PYG{p}{)}
\end{sphinxVerbatim}

\sphinxAtStartPar
See the matplotlib \sphinxhref{https://matplotlib.org/users/pyplot\_tutorial.html}{pyplot tutorial} and
the \sphinxhref{https://matplotlib.org/gallery.html}{gallery} for
lots of examples of matplotlib plots.


\section{Inheritance diagrams}
\label{\detokenize{chapter2:inheritance-diagrams}}
\sphinxAtStartPar
Inheritance diagrams can be inserted directly into the document by
providing a list of class or module names to the
\sphinxcode{\sphinxupquote{inheritance\sphinxhyphen{}diagram}} directive.

\sphinxAtStartPar
For example:

\begin{sphinxVerbatim}[commandchars=\\\{\}]
\PYG{o}{.}\PYG{o}{.} \PYG{n}{inheritance}\PYG{o}{\PYGZhy{}}\PYG{n}{diagram}\PYG{p}{:}\PYG{p}{:} \PYG{n}{codecs}
\end{sphinxVerbatim}

\sphinxAtStartPar
produces:

\sphinxAtStartPar
//.. inheritance\sphinxhyphen{}diagram:: codecs

\sphinxAtStartPar
//
matplotlib aware ipython sessions into your rest docs with multiline
and doctest support.


\section{This file}
\label{\detokenize{chapter2:this-file}}\label{\detokenize{chapter2:extensions-literal}}
\sphinxAtStartPar
//.. literalinclude:: extensions.rst


\chapter{Markdown Cheatsheet}
\label{\detokenize{chapter3:markdown-cheatsheet}}\label{\detokenize{chapter3::doc}}\begin{quote}

\sphinxAtStartPar
\sphinxurl{https://github.com/tchapi/markdown-cheatsheet}
\end{quote}


\bigskip\hrule\bigskip



\chapter{Heading 1}
\label{\detokenize{chapter3:heading-1}}
\sphinxAtStartPar
Markup :  \# Heading 1 \#

\sphinxAtStartPar
\sphinxhyphen{}OR\sphinxhyphen{}

\sphinxAtStartPar
Markup :  ============= (below H1 text)


\section{Heading 2}
\label{\detokenize{chapter3:heading-2}}
\sphinxAtStartPar
Markup :  \#\# Heading 2 \#\#

\sphinxAtStartPar
\sphinxhyphen{}OR\sphinxhyphen{}

\sphinxAtStartPar
Markup: ————— (below H2 text)


\subsection{Heading 3}
\label{\detokenize{chapter3:heading-3}}
\sphinxAtStartPar
Markup :  \#\#\# Heading 3 \#\#\#


\subsubsection{Heading 4}
\label{\detokenize{chapter3:heading-4}}
\sphinxAtStartPar
Markup :  \#\#\#\# Heading 4 \#\#\#\#

\sphinxAtStartPar
Common text

\begin{sphinxVerbatim}[commandchars=\\\{\}]
\PYG{n}{Markup} \PYG{p}{:}  \PYG{n}{Common} \PYG{n}{text}
\end{sphinxVerbatim}

\sphinxAtStartPar
\sphinxstyleemphasis{Emphasized text}

\begin{sphinxVerbatim}[commandchars=\\\{\}]
\PYG{n}{Markup} \PYG{p}{:}  \PYG{n}{\PYGZus{}Emphasized} \PYG{n}{text\PYGZus{}} \PYG{o+ow}{or} \PYG{o}{*}\PYG{n}{Emphasized} \PYG{n}{text}\PYG{o}{*}
\end{sphinxVerbatim}

\sphinxAtStartPar
\textasciitilde{}\textasciitilde{}Strikethrough text\textasciitilde{}\textasciitilde{}

\begin{sphinxVerbatim}[commandchars=\\\{\}]
\PYG{n}{Markup} \PYG{p}{:}  \PYG{o}{\PYGZti{}}\PYG{o}{\PYGZti{}}\PYG{n}{Strikethrough} \PYG{n}{text}\PYG{o}{\PYGZti{}}\PYG{o}{\PYGZti{}}
\end{sphinxVerbatim}

\sphinxAtStartPar
\sphinxstylestrong{Strong text}

\begin{sphinxVerbatim}[commandchars=\\\{\}]
\PYG{n}{Markup} \PYG{p}{:}  \PYG{n}{\PYGZus{}\PYGZus{}Strong} \PYG{n}{text\PYGZus{}\PYGZus{}} \PYG{o+ow}{or} \PYG{o}{*}\PYG{o}{*}\PYG{n}{Strong} \PYG{n}{text}\PYG{o}{*}\PYG{o}{*}
\end{sphinxVerbatim}

\sphinxAtStartPar
\sphinxstyleemphasis{\sphinxstylestrong{Strong emphasized text}}

\begin{sphinxVerbatim}[commandchars=\\\{\}]
\PYG{n}{Markup} \PYG{p}{:}  \PYG{n}{\PYGZus{}\PYGZus{}\PYGZus{}Strong} \PYG{n}{emphasized} \PYG{n}{text\PYGZus{}\PYGZus{}\PYGZus{}} \PYG{o+ow}{or} \PYG{o}{*}\PYG{o}{*}\PYG{o}{*}\PYG{n}{Strong} \PYG{n}{emphasized} \PYG{n}{text}\PYG{o}{*}\PYG{o}{*}\PYG{o}{*}
\end{sphinxVerbatim}

\sphinxAtStartPar
\sphinxhref{http://www.google.fr/}{Named Link} and http://www.google.fr/ or \sphinxurl{http://example.com/}

\begin{sphinxVerbatim}[commandchars=\\\{\}]
\PYG{n}{Markup} \PYG{p}{:}  \PYG{p}{[}\PYG{n}{Named} \PYG{n}{Link}\PYG{p}{]}\PYG{p}{(}\PYG{n}{http}\PYG{p}{:}\PYG{o}{/}\PYG{o}{/}\PYG{n}{www}\PYG{o}{.}\PYG{n}{google}\PYG{o}{.}\PYG{n}{fr}\PYG{o}{/} \PYG{l+s+s2}{\PYGZdq{}}\PYG{l+s+s2}{Named link title}\PYG{l+s+s2}{\PYGZdq{}}\PYG{p}{)} \PYG{o+ow}{and} \PYG{n}{http}\PYG{p}{:}\PYG{o}{/}\PYG{o}{/}\PYG{n}{www}\PYG{o}{.}\PYG{n}{google}\PYG{o}{.}\PYG{n}{fr}\PYG{o}{/} \PYG{o+ow}{or} \PYG{o}{\PYGZlt{}}\PYG{n}{http}\PYG{p}{:}\PYG{o}{/}\PYG{o}{/}\PYG{n}{example}\PYG{o}{.}\PYG{n}{com}\PYG{o}{/}\PYG{o}{\PYGZgt{}}
\end{sphinxVerbatim}

\sphinxAtStartPar
{\hyperref[\detokenize{chapter3:heading-1}]{\emph{heading\sphinxhyphen{}1}}}

\begin{sphinxVerbatim}[commandchars=\\\{\}]
\PYG{n}{Markup}\PYG{p}{:} \PYG{p}{[}\PYG{n}{heading}\PYG{o}{\PYGZhy{}}\PYG{l+m+mi}{1}\PYG{p}{]}\PYG{p}{(}\PYG{c+c1}{\PYGZsh{}heading\PYGZhy{}1 \PYGZdq{}Goto heading\PYGZhy{}1\PYGZdq{})}
\end{sphinxVerbatim}

\sphinxAtStartPar
Table, like this one :

\sphinxAtStartPar
First Header  | Second Header
————\sphinxhyphen{} | ————\sphinxhyphen{}
Content Cell  | Content Cell
Content Cell  | Content Cell

\begin{sphinxVerbatim}[commandchars=\\\{\}]
\PYG{n}{First} \PYG{n}{Header}  \PYG{o}{|} \PYG{n}{Second} \PYG{n}{Header}
\PYG{o}{\PYGZhy{}}\PYG{o}{\PYGZhy{}}\PYG{o}{\PYGZhy{}}\PYG{o}{\PYGZhy{}}\PYG{o}{\PYGZhy{}}\PYG{o}{\PYGZhy{}}\PYG{o}{\PYGZhy{}}\PYG{o}{\PYGZhy{}}\PYG{o}{\PYGZhy{}}\PYG{o}{\PYGZhy{}}\PYG{o}{\PYGZhy{}}\PYG{o}{\PYGZhy{}}\PYG{o}{\PYGZhy{}} \PYG{o}{|} \PYG{o}{\PYGZhy{}}\PYG{o}{\PYGZhy{}}\PYG{o}{\PYGZhy{}}\PYG{o}{\PYGZhy{}}\PYG{o}{\PYGZhy{}}\PYG{o}{\PYGZhy{}}\PYG{o}{\PYGZhy{}}\PYG{o}{\PYGZhy{}}\PYG{o}{\PYGZhy{}}\PYG{o}{\PYGZhy{}}\PYG{o}{\PYGZhy{}}\PYG{o}{\PYGZhy{}}\PYG{o}{\PYGZhy{}}
\PYG{n}{Content} \PYG{n}{Cell}  \PYG{o}{|} \PYG{n}{Content} \PYG{n}{Cell}
\PYG{n}{Content} \PYG{n}{Cell}  \PYG{o}{|} \PYG{n}{Content} \PYG{n}{Cell}
\end{sphinxVerbatim}

\sphinxAtStartPar
\sphinxcode{\sphinxupquote{code()}}

\begin{sphinxVerbatim}[commandchars=\\\{\}]
Markup :  `code()`
\end{sphinxVerbatim}

\begin{sphinxVerbatim}[commandchars=\\\{\}]
    \PYG{k+kd}{var} \PYG{n+nx}{specificLanguage\PYGZus{}code} \PYG{o}{=} 
    \PYG{p}{\PYGZob{}}
        \PYG{l+s+s2}{\PYGZdq{}data\PYGZdq{}}\PYG{o}{:} \PYG{p}{\PYGZob{}}
            \PYG{l+s+s2}{\PYGZdq{}lookedUpPlatform\PYGZdq{}}\PYG{o}{:} \PYG{l+m+mf}{1}\PYG{p}{,}
            \PYG{l+s+s2}{\PYGZdq{}query\PYGZdq{}}\PYG{o}{:} \PYG{l+s+s2}{\PYGZdq{}Kasabian+Test+Transmission\PYGZdq{}}\PYG{p}{,}
            \PYG{l+s+s2}{\PYGZdq{}lookedUpItem\PYGZdq{}}\PYG{o}{:} \PYG{p}{\PYGZob{}}
                \PYG{l+s+s2}{\PYGZdq{}name\PYGZdq{}}\PYG{o}{:} \PYG{l+s+s2}{\PYGZdq{}Test Transmission\PYGZdq{}}\PYG{p}{,}
                \PYG{l+s+s2}{\PYGZdq{}artist\PYGZdq{}}\PYG{o}{:} \PYG{l+s+s2}{\PYGZdq{}Kasabian\PYGZdq{}}\PYG{p}{,}
                \PYG{l+s+s2}{\PYGZdq{}album\PYGZdq{}}\PYG{o}{:} \PYG{l+s+s2}{\PYGZdq{}Kasabian\PYGZdq{}}\PYG{p}{,}
                \PYG{l+s+s2}{\PYGZdq{}picture\PYGZdq{}}\PYG{o}{:} \PYG{k+kc}{null}\PYG{p}{,}
                \PYG{l+s+s2}{\PYGZdq{}link\PYGZdq{}}\PYG{o}{:} \PYG{l+s+s2}{\PYGZdq{}http://open.spotify.com/track/5jhJur5n4fasblLSCOcrTp\PYGZdq{}}
            \PYG{p}{\PYGZcb{}}
        \PYG{p}{\PYGZcb{}}
    \PYG{p}{\PYGZcb{}}
\end{sphinxVerbatim}

\begin{sphinxVerbatim}[commandchars=\\\{\}]
Markup : ```javascript
         ```
\end{sphinxVerbatim}
\begin{itemize}
\item {} 
\sphinxAtStartPar
Bullet list
\begin{itemize}
\item {} 
\sphinxAtStartPar
Nested bullet
\begin{itemize}
\item {} 
\sphinxAtStartPar
Sub\sphinxhyphen{}nested bullet etc

\end{itemize}

\end{itemize}

\item {} 
\sphinxAtStartPar
Bullet list item 2

\end{itemize}

\begin{sphinxVerbatim}[commandchars=\\\{\}]
 \PYG{n}{Markup} \PYG{p}{:} \PYG{o}{*} \PYG{n}{Bullet} \PYG{n+nb}{list}
              \PYG{o}{*} \PYG{n}{Nested} \PYG{n}{bullet}
                  \PYG{o}{*} \PYG{n}{Sub}\PYG{o}{\PYGZhy{}}\PYG{n}{nested} \PYG{n}{bullet} \PYG{n}{etc}
          \PYG{o}{*} \PYG{n}{Bullet} \PYG{n+nb}{list} \PYG{n}{item} \PYG{l+m+mi}{2}
\end{sphinxVerbatim}
\begin{enumerate}
\sphinxsetlistlabels{\arabic}{enumi}{enumii}{}{.}%
\item {} 
\sphinxAtStartPar
A numbered list
\begin{enumerate}
\sphinxsetlistlabels{\arabic}{enumii}{enumiii}{}{.}%
\item {} 
\sphinxAtStartPar
A nested numbered list

\item {} 
\sphinxAtStartPar
Which is numbered

\end{enumerate}

\item {} 
\sphinxAtStartPar
Which is numbered

\end{enumerate}

\begin{sphinxVerbatim}[commandchars=\\\{\}]
 \PYG{n}{Markup} \PYG{p}{:} \PYG{l+m+mf}{1.} \PYG{n}{A} \PYG{n}{numbered} \PYG{n+nb}{list}
              \PYG{l+m+mf}{1.} \PYG{n}{A} \PYG{n}{nested} \PYG{n}{numbered} \PYG{n+nb}{list}
              \PYG{l+m+mf}{2.} \PYG{n}{Which} \PYG{o+ow}{is} \PYG{n}{numbered}
          \PYG{l+m+mf}{2.} \PYG{n}{Which} \PYG{o+ow}{is} \PYG{n}{numbered}
\end{sphinxVerbatim}
\begin{itemize}
\item {} 
\sphinxAtStartPar
{[} {]} An uncompleted task

\item {} 
\sphinxAtStartPar
{[}x{]} A completed task

\end{itemize}

\begin{sphinxVerbatim}[commandchars=\\\{\}]
 \PYG{n}{Markup} \PYG{p}{:} \PYG{o}{\PYGZhy{}} \PYG{p}{[} \PYG{p}{]} \PYG{n}{An} \PYG{n}{uncompleted} \PYG{n}{task}
          \PYG{o}{\PYGZhy{}} \PYG{p}{[}\PYG{n}{x}\PYG{p}{]} \PYG{n}{A} \PYG{n}{completed} \PYG{n}{task}
\end{sphinxVerbatim}
\begin{quote}

\sphinxAtStartPar
Blockquote
\begin{quote}

\sphinxAtStartPar
Nested blockquote
\end{quote}
\end{quote}

\begin{sphinxVerbatim}[commandchars=\\\{\}]
\PYG{n}{Markup} \PYG{p}{:}  \PYG{o}{\PYGZgt{}} \PYG{n}{Blockquote}
          \PYG{o}{\PYGZgt{}\PYGZgt{}} \PYG{n}{Nested} \PYG{n}{Blockquote}
\end{sphinxVerbatim}

\sphinxAtStartPar
\sphinxstyleemphasis{Horizontal line :}


\bigskip\hrule\bigskip


\begin{sphinxVerbatim}[commandchars=\\\{\}]
\PYG{n}{Markup} \PYG{p}{:}  \PYG{o}{\PYGZhy{}} \PYG{o}{\PYGZhy{}} \PYG{o}{\PYGZhy{}} \PYG{o}{\PYGZhy{}}
\end{sphinxVerbatim}

\sphinxAtStartPar
\sphinxstyleemphasis{Image with alt :}

\sphinxAtStartPar


\begin{sphinxVerbatim}[commandchars=\\\{\}]
Markup : ![picture alt](http://via.placeholder.com/200x150 \PYGZdq{}Title is optional\PYGZdq{})
\end{sphinxVerbatim}

\sphinxAtStartPar
Foldable text:



\begin{sphinxVerbatim}[commandchars=\\\{\}]
\PYG{n}{Markup} \PYG{p}{:} \PYG{o}{\PYGZlt{}}\PYG{n}{details}\PYG{o}{\PYGZgt{}}
           \PYG{o}{\PYGZlt{}}\PYG{n}{summary}\PYG{o}{\PYGZgt{}}\PYG{n}{Title} \PYG{l+m+mi}{1}\PYG{o}{\PYGZlt{}}\PYG{o}{/}\PYG{n}{summary}\PYG{o}{\PYGZgt{}}
           \PYG{o}{\PYGZlt{}}\PYG{n}{p}\PYG{o}{\PYGZgt{}}\PYG{n}{Content} \PYG{l+m+mi}{1} \PYG{n}{Content} \PYG{l+m+mi}{1} \PYG{n}{Content} \PYG{l+m+mi}{1} \PYG{n}{Content} \PYG{l+m+mi}{1} \PYG{n}{Content} \PYG{l+m+mi}{1}\PYG{o}{\PYGZlt{}}\PYG{o}{/}\PYG{n}{p}\PYG{o}{\PYGZgt{}}
         \PYG{o}{\PYGZlt{}}\PYG{o}{/}\PYG{n}{details}\PYG{o}{\PYGZgt{}}
\end{sphinxVerbatim}

\begin{sphinxVerbatim}[commandchars=\\\{\}]
\PYG{p}{\PYGZlt{}}\PYG{n+nt}{h3}\PYG{p}{\PYGZgt{}}HTML\PYG{p}{\PYGZlt{}}\PYG{p}{/}\PYG{n+nt}{h3}\PYG{p}{\PYGZgt{}}
\PYG{p}{\PYGZlt{}}\PYG{n+nt}{p}\PYG{p}{\PYGZgt{}} Some HTML code here \PYG{p}{\PYGZlt{}}\PYG{p}{/}\PYG{n+nt}{p}\PYG{p}{\PYGZgt{}}
\end{sphinxVerbatim}

\sphinxAtStartPar
Hotkey:

\sphinxAtStartPar
⌘F

\sphinxAtStartPar
⇧⌘F

\begin{sphinxVerbatim}[commandchars=\\\{\}]
Markup : \PYGZlt{}kbd\PYGZgt{}⌘F\PYGZlt{}/kbd\PYGZgt{}
\end{sphinxVerbatim}

\sphinxAtStartPar
Hotkey list:

\sphinxAtStartPar
| Key | Symbol |
| — | — |
| Option | ⌥ |
| Control | ⌃ |
| Command | ⌘ |
| Shift | ⇧ |
| Caps Lock | ⇪ |
| Tab | ⇥ |
| Esc | ⎋ |
| Power | ⌽ |
| Return | ↩ |
| Delete | ⌫ |
| Up | ↑ |
| Down | ↓ |
| Left | ← |
| Right | \(\rightarrow\) |

\sphinxAtStartPar
Emoji:

\sphinxAtStartPar
:exclamation: Use emoji icons to enhance text. :+1:  Look up emoji codes at \sphinxhref{http://emoji-cheat-sheet.com/}{emoji\sphinxhyphen{}cheat\sphinxhyphen{}sheet.com}

\begin{sphinxVerbatim}[commandchars=\\\{\}]
\PYG{n}{Markup} \PYG{p}{:} \PYG{n}{Code} \PYG{n}{appears} \PYG{n}{between} \PYG{n}{colons} \PYG{p}{:}\PYG{n}{EMOJICODE}\PYG{p}{:}
\end{sphinxVerbatim}
\phantomsection\label{\detokenize{rst-text:rst-text}}
\sphinxAtStartPar
\sphinxstyleemphasis{Welcome!}

\sphinxAtStartPar
\sphinxstylestrong{let’s learn reStructturedText}


\chapter{一级标题}
\label{\detokenize{rst-text:id1}}\label{\detokenize{rst-text::doc}}

\section{二级标题}
\label{\detokenize{rst-text:id2}}

\subsection{三级级标题}
\label{\detokenize{rst-text:id3}}

\chapter{reStructturedText 语法学习}
\label{\detokenize{rst-text:restructturedtext}}
\sphinxAtStartPar
分隔符上


\bigskip\hrule\bigskip


\sphinxAtStartPar
分隔符下


\section{标注}
\label{\detokenize{rst-text:id4}}
\sphinxAtStartPar
let’s learn label %
\begin{footnote}[1]\sphinxAtStartFootnote
标注的使用
%
\end{footnote} ! you’ll see a footnote %
\begin{footnote}[2]\sphinxAtStartFootnote
这是第二条脚注.
%
\end{footnote} .


\section{列表}
\label{\detokenize{rst-text:list}}\label{\detokenize{rst-text:id7}}\begin{itemize}
\item {} 
\sphinxAtStartPar
无序列表

\item {} 
\sphinxAtStartPar
第一项

\item {} 
\sphinxAtStartPar
第二项

\item {} 
\sphinxAtStartPar
第三项

\end{itemize}
\begin{enumerate}
\sphinxsetlistlabels{\arabic}{enumi}{enumii}{}{.}%
\item {} 
\sphinxAtStartPar
有序列表

\item {} 
\sphinxAtStartPar
第一项

\item {} 
\sphinxAtStartPar
第二项

\item {} 
\sphinxAtStartPar
第三项

\end{enumerate}
\begin{description}
\item[{定义列表}] \leavevmode\begin{quote}\begin{description}
\item[{字段列表}] \leavevmode
\sphinxAtStartPar
第一项

\end{description}\end{quote}

\item[{how}] \leavevmode\begin{quote}\begin{description}
\item[{第二项}] \leavevmode
\sphinxAtStartPar
第二项

\end{description}\end{quote}

\end{description}
\begin{enumerate}
\sphinxsetlistlabels{\arabic}{enumi}{enumii}{}{.}%
\item {} 
\sphinxAtStartPar
列表嵌套

\item {} 
\sphinxAtStartPar
父列表第一项

\item {} 
\sphinxAtStartPar
父列表第二项

\end{enumerate}
\begin{itemize}
\item {} 
\sphinxAtStartPar
子列表第一项

\item {} 
\sphinxAtStartPar
子列表第二项

\end{itemize}
\begin{enumerate}
\sphinxsetlistlabels{\arabic}{enumi}{enumii}{}{.}%
\setcounter{enumi}{3}
\item {} 
\sphinxAtStartPar
父列表第三项

\end{enumerate}


\section{代码}
\label{\detokenize{rst-text:id8}}
\sphinxAtStartPar
\sphinxtitleref{code}

\sphinxAtStartPar
第一段文本

\begin{sphinxVerbatim}[commandchars=\\\{\}]
代码区块演示
if(1 == 1)\PYGZob{}
     \PYGZdl{}joke = \PYGZdq{}Life is short, not int.\PYGZdq{};
 \PYGZcb{}
\end{sphinxVerbatim}

\sphinxAtStartPar
代码高亮展示

\begin{sphinxVerbatim}[commandchars=\\\{\}]
\PYG{o}{\PYGZlt{}}\PYG{o}{?}\PYG{n}{java}
\PYG{n+nf}{if}\PYG{p}{(}\PYG{l+m+mi}{1} \PYG{o}{=}\PYG{o}{=} \PYG{l+m+mi}{1}\PYG{p}{)}\PYG{p}{\PYGZob{}}
    \PYG{n}{\PYGZdl{}joke} \PYG{o}{=} \PYG{l+s}{\PYGZdq{}}\PYG{l+s}{Life is short, not int.}\PYG{l+s}{\PYGZdq{}}\PYG{p}{;}
\PYG{p}{\PYGZcb{}}
\PYG{o}{?}\PYG{o}{\PYGZgt{}}
\end{sphinxVerbatim}

\begin{sphinxVerbatim}[commandchars=\\\{\}]
\PYG{n}{引用文本}
\end{sphinxVerbatim}


\section{链接}
\label{\detokenize{rst-text:id9}}
\sphinxAtStartPar
\sphinxstylestrong{外部链接}

\sphinxAtStartPar
\sphinxhref{http://www.sphinx-doc.org/en/master/}{Sphinx官网}

\sphinxAtStartPar
\sphinxhref{https://dac-tut.readthedocs.io/zh\_CN/latest/rst-guide.html}{Sphinx使用教程}

\sphinxAtStartPar
\sphinxtitleref{Markdown基本语法 \textless{}https://markdown.com.cn/basic\sphinxhyphen{}syntax/emphasis.html\textgreater{}}

\sphinxAtStartPar
\sphinxstylestrong{内部链接}

\sphinxAtStartPar
{\hyperref[\detokenize{rst-text:list}]{\sphinxcrossref{\DUrole{std,std-ref}{列表}}}}.


\section{图片}
\label{\detokenize{rst-text:id11}}
\noindent\sphinxincludegraphics{{rst-insert-sphinx}.jpg}


\section{表格}
\label{\detokenize{rst-text:id12}}

\begin{savenotes}\sphinxattablestart
\centering
\begin{tabulary}{\linewidth}[t]{|T|T|T|}
\hline
\sphinxstyletheadfamily 
\sphinxAtStartPar
A
&\sphinxstyletheadfamily 
\sphinxAtStartPar
B
&\sphinxstyletheadfamily 
\sphinxAtStartPar
A and B
\\
\hline
\sphinxAtStartPar
False
&
\sphinxAtStartPar
False
&
\sphinxAtStartPar
False
\\
\hline
\sphinxAtStartPar
True
&
\sphinxAtStartPar
False
&
\sphinxAtStartPar
False
\\
\hline
\sphinxAtStartPar
False
&
\sphinxAtStartPar
True
&
\sphinxAtStartPar
False
\\
\hline
\sphinxAtStartPar
True
&
\sphinxAtStartPar
True
&
\sphinxAtStartPar
True
\\
\hline
\end{tabulary}
\par
\sphinxattableend\end{savenotes}

\sphinxAtStartPar
\sphinxstylestrong{单元格换行以及单元格置空}


\begin{savenotes}\sphinxattablestart
\centering
\begin{tabular}[t]{|*{2}{\X{1}{2}|}}
\hline
\sphinxstyletheadfamily 
\sphinxAtStartPar
col 1
&\sphinxstyletheadfamily 
\sphinxAtStartPar
col 2
\\
\hline
\sphinxAtStartPar
1
&
\sphinxAtStartPar
Second column of row 1.
\\
\hline
\sphinxAtStartPar
2
&
\sphinxAtStartPar
Second column of row 2.
Second line of paragraph.
\\
\hline
\sphinxAtStartPar
3
&\begin{itemize}
\item {} 
\sphinxAtStartPar
Second column of row 3.

\item {} 
\sphinxAtStartPar
Second item in bullet
list (row 3, column 2).

\end{itemize}
\\
\hline
\sphinxAtStartPar

&
\sphinxAtStartPar
Row 4; column 1 will be empty.
\\
\hline
\end{tabular}
\par
\sphinxattableend\end{savenotes}


\subsection{网格表格}
\label{\detokenize{rst-text:id13}}
\sphinxAtStartPar
网格表格可以自定义表格的边框,更灵活,但绘制相对复杂。构成网格表格的标记有以下几种:
\begin{itemize}
\item {} 
\sphinxAtStartPar
“\sphinxhyphen{}“用于绘制横线,分隔各行;

\item {} 
\sphinxAtStartPar
“=”用于分隔标题与表格主体,但标题可有可无,视情况而定;

\item {} 
\sphinxAtStartPar
“|”用于绘制竖线,分隔各列;

\item {} 
\sphinxAtStartPar
“+”用在行与列的交界处。

\end{itemize}


\begin{savenotes}\sphinxattablestart
\centering
\begin{tabular}[t]{|*{4}{\X{1}{4}|}}
\hline
\sphinxstyletheadfamily 
\sphinxAtStartPar
Header row, column 1
&\sphinxstyletheadfamily 
\sphinxAtStartPar
Header 2
&\sphinxstyletheadfamily 
\sphinxAtStartPar
Header 3
&\sphinxstyletheadfamily 
\sphinxAtStartPar
Header 4
\\
\hline
\sphinxAtStartPar
body row 1, column 1
&
\sphinxAtStartPar
column 2
&
\sphinxAtStartPar
column 3
&
\sphinxAtStartPar
column 4
\\
\hline
\sphinxAtStartPar
body row 2
&\sphinxstartmulticolumn{3}%
\begin{varwidth}[t]{\sphinxcolwidth{3}{4}}
\sphinxAtStartPar
Cells may span columns.
\par
\vskip-\baselineskip\vbox{\hbox{\strut}}\end{varwidth}%
\sphinxstopmulticolumn
\\
\hline
\sphinxAtStartPar
body row 3
&\sphinxmultirow{2}{12}{%
\begin{varwidth}[t]{\sphinxcolwidth{1}{4}}
\sphinxAtStartPar
Cells may
span rows.
\par
\vskip-\baselineskip\vbox{\hbox{\strut}}\end{varwidth}%
}%
&\sphinxstartmulticolumn{2}%
\sphinxmultirow{2}{13}{%
\begin{varwidth}[t]{\sphinxcolwidth{2}{4}}
\begin{itemize}
\item {} 
\sphinxAtStartPar
Table cells

\item {} 
\sphinxAtStartPar
contain

\item {} 
\sphinxAtStartPar
body elements.

\end{itemize}
\par
\vskip-\baselineskip\vbox{\hbox{\strut}}\end{varwidth}%
}%
\sphinxstopmulticolumn
\\
\cline{1-1}
\sphinxAtStartPar
body row 4
&\sphinxtablestrut{12}&\multicolumn{2}{l|}{\sphinxtablestrut{13}}\\
\hline
\end{tabular}
\par
\sphinxattableend\end{savenotes}


\subsection{列表表格}
\label{\detokenize{rst-text:id14}}

\begin{savenotes}\sphinxattablestart
\raggedleft
\begin{tabulary}{\linewidth}[t]{|T|T|T|}
\hline

\sphinxAtStartPar
单行代码
&
\sphinxAtStartPar
代码区块
&
\sphinxAtStartPar
代码高亮
\\
\hline
\sphinxAtStartPar
简单表格
&
\sphinxAtStartPar
网格表格
&
\sphinxAtStartPar
列表表格
\\
\hline
\sphinxAtStartPar
外部链接
&
\sphinxAtStartPar
内部链接
&\\
\hline
\end{tabulary}
\par
\sphinxattableend\end{savenotes}


\chapter{Dock like code}
\label{\detokenize{Dock-like-code:dock-like-code}}\label{\detokenize{Dock-like-code::doc}}

\chapter{1.Dock like code}
\label{\detokenize{Dock-like-code:id1}}

\section{1.1 What Dock like code}
\label{\detokenize{Dock-like-code:what-dock-like-code}}
\sphinxAtStartPar
Docs Like Code 不是一个工具,而是一种理念。这种理念决定了一套写文档的完整流程和工具链,即:写文档与写代码的流程和工具链保持一致。


\section{Why Dock like code}
\label{\detokenize{Dock-like-code:why-dock-like-code}}
\sphinxAtStartPar
据 Stack Overflow(著名开发者问答社区)2016 年的调查 {[}2{]} 显示,文档质量差是软件工程师们在工作中最常抱怨的三大问题之一。\sphinxstylestrong{文档不完整、文档找不到、文档过时}逐渐成为软件工程师们面临的“三堵高墙”。

\sphinxAtStartPar
\sphinxstylestrong{文档必须紧跟代码,迅速迭代优化}


\section{Docs Like Code 运动}
\label{\detokenize{Dock-like-code:docs-like-code}}
\sphinxAtStartPar
Write the Docs 2014 会议 {[}3{]} 上,来自 Twitter 的 Simeon Franklin 和 Marko Gargenta 做了 Docs Like Code 的实践分享,台下来自 Google 的 Riona MacNamara 备受鼓舞。
次年 Write the Docs 会议 {[}4{]} 上,Riona 分享了在 Google 进行 Docs Like Code 实践的成功经验。

\sphinxAtStartPar
Docs Like Code 慢慢成为了 Google 司软件文档开发的一项事实标准。


\section{Docs Like Code 的优势}
\label{\detokenize{Dock-like-code:id2}}\begin{enumerate}
\sphinxsetlistlabels{\arabic}{enumi}{enumii}{}{.}%
\item {} 
\sphinxAtStartPar
\sphinxstylestrong{文档任务能更早地融入到软件版本发布的流程}

\item {} 
\sphinxAtStartPar
\sphinxstylestrong{文档持续、即时发布}

\end{enumerate}


\chapter{Apifox}
\label{\detokenize{Apifox:apifox}}\label{\detokenize{Apifox::doc}}

\chapter{Indices and tables}
\label{\detokenize{index:indices-and-tables}}\begin{itemize}
\item {} 
\sphinxAtStartPar
\DUrole{xref,std,std-ref}{genindex}

\item {} 
\sphinxAtStartPar
\DUrole{xref,std,std-ref}{modindex}

\item {} 
\sphinxAtStartPar
\DUrole{xref,std,std-ref}{search}

\end{itemize}



\renewcommand{\indexname}{Index}
\printindex
\end{document}